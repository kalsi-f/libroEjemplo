\documentclass[12pt, letterpaper]{article}

% --- Paquetes de configuración ---
\usepackage[utf8]{inputenc}    % Para tildes y eñes
\usepackage[spanish]{babel}    % Idioma español
\usepackage{amsmath, amssymb}  % Símbolos matemáticos
\usepackage{graphicx}          % Para insertar imágenes
\usepackage{geometry}          % Para márgenes
\geometry{margin=2.5cm}

% --- Título y Autor ---
\title{Ejemplo de archivo en Latex\LaTeX}
\author{Tu Nombre}
\date{\today}

\begin{document}

\maketitle % Crea la portada automáticamente

\section{Introducción}
Este es un archivo de \LaTeX{} completo. A diferencia de Markdown, aquí tenemos control total sobre el formato. 

\section{Matemáticas Interactivas}
Podemos escribir ecuaciones complejas de forma profesional. Por ejemplo, la famosa ecuación de Einstein:
\begin{equation}
    E = mc^2
\end{equation}

También podemos usar macros personalizadas. Si definimos un comando para los números reales, se vería así: $\mathbb{R}$.

\section{Listas y Formato}
Podemos crear listas fácilmente:
\begin{itemize}
    \item \textbf{Negrita}: Para resaltar ideas.
    \item \textit{Cursiva}: Para énfasis.
    \item \texttt{Código}: Para texto monoespaciado.
\end{itemize}

\section{Conclusión}
Este archivo está listo para ser compilado en un PDF. Si estás usando Jupyter Book, puedes importar este contenido o usarlo como base para tu preámbulo.

\end{document}